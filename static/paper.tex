\documentclass[letterpaper,12pt,leqno]{article}
\usepackage{paper}
\bibliographystyle{bibliography}
\hypersetup{pdftitle={Paper Example}}
% Enter permanent URL to paper
\available{https://github.com/pmichaillat/latex-paper}
% Enter BibTeX file with references here:
\newcommand{\bib}{bibliography.bib}
% Enter PDF file with figures here:
\newcommand{\pdf}{figures.pdf}


\begin{document}

\title{Paper Title}
\author{Author 1, Author 2
\thanks{Author 1: University 1. Author 2: University 2. We thank many colleagues for helpful comments and discussions. This work was supported by the a generous grant [grant number]; and another generous foundation.}}
\date{Month Year}                       

\begin{titlepage}\maketitle

Lorem ipsum dolor sit amet, consectetur adipiscing elit, sed do eiusmod tempor incididunt ut labore et dolore magna aliqua. Ut enim ad minim veniam, quis nostrud exercitation ullamco laboris nisi ut aliquip ex ea commodo consequat. Duis aute irure dolor in reprehenderit in voluptate velit esse cillum dolore eu fugiat nulla pariatur. Excepteur sint occaecat cupidatat non proident, sunt in culpa qui officia deserunt mollit anim id est laborum.

\end{titlepage}\section{Introduction}\label{s:introduction}
 
\paragraph{Research question} Lorem ipsum dolor sit amet, consectetur adipiscing elit, sed do eiusmod tempor incididunt ut labore et dolore magna aliqua. Ut enim ad minim veniam, quis nostrud exercitation ullamco laboris nisi ut aliquip ex ea commodo consequat. Duis aute irure dolor in reprehenderit in voluptate velit esse cillum dolore eu fugiat nulla pariatur.\footnote{Excepteur sint, sunt in culpa qui officia deserunt mollit anim id est laborum.}

\paragraph{Positioning in the literature} Lorem ipsum dolor sit amet \citep{MS15}, consectetur adipiscing elit \citep[p. 1305]{MS19}, sed do eiusmod tempor incididunt ut labore et dolore magna aliqua . \citet{M12,M14} enim ad minim veniam, quis nostrud exercitation ullamco laboris nisi ut aliquip ex ea commodo consequat. Duis aute irure dolor in reprehenderit in voluptate velit esse cillum dolore eu fugiat nulla pariatur.  Duis aute irure dolor in reprehenderit in voluptate velit esse cillum dolore eu fugiat nulla pariatur. \citeauthor{EMM21} aute irure dolor in reprehenderit in voluptate velit esse cillum dolore eu fugiat nulla pariatur.\footnote{In reprehenderit in eu fugiat nulla pariatur \citep{LMS18a}.}

\paragraph{Answer to the question} Lorem ipsum dolor sit amet, consectetur adipiscing elit, sed do eiusmod tempor incididunt ut labore et dolore magna aliqua. Ut enim ad minim veniam, quis nostrud exercitation ullamco laboris nisi ut aliquip ex ea commodo consequat.\footnote{Sunt in culpa qui officia deserunt mollit anim id est laborum. Excepteur sint occaecat cupidatat non proident. Sunt in culpa qui officia deserunt mollit anim id est laborum. Excepteur sint occaecat cupidatat non proident.} Duis aute irure dolor in reprehenderit in voluptate velit esse cillum dolore eu fugiat nulla pariatur. 

\paragraph{Another paragraph} Sed ut perspiciatis, unde omnis iste natus error sit voluptatem accusantium doloremque laudantium, totam rem aperiam eaque ipsa, quae ab illo inventore veritatis et quasi architecto beatae vitae dicta sunt, explicabo. Nemo enim ipsam voluptatem, quia voluptas sit, aspernatur aut odit aut fugit, sed quia consequuntur magni dolores eos, qui ratione voluptatem sequi nesciunt, neque porro quisquam est, qui dolorem ipsum, quia dolor sit amet consectetur adipiscing velit, sed quia non numquam eius modi tempora incidunt, ut labore et dolore magnam aliquam quaerat voluptatem.

\section{Section}\label{s:section}

Ut aut reiciendis voluptatibus maiores alias consequatur aut perferendis doloribus asperiores repellat. 

\subsection{A subsection with lists}\label{s:lists}

At vero eos et accusamus et iusto odio dignissimos ducimus, qui blanditiis praesentium voluptatum deleniti atque corrupti, quos dolores et quas molestias excepturi sint.

Oobcaecati cupiditate non provident, similique sunt in culpa, qui officia deserunt mollitia animi, id est laborum et dolorum fuga. Here is a list:
\begin{itemize}
 	\item Et harum quidem rerum facilis est et expedita distinctio.
 	\item Nam libero tempore, cum soluta nobis est eligendi optio.
 	\item emporibus autem quibusdam et aut officiis debitis aut rerum necessitatibus saepe eveniet.
 		\begin{itemize}
 			\item cumque nihil impedit
 			\item quo minus id, quod maxime placeat
 			\item facere possimus, omnis voluptas assumenda est
 		\end{itemize}
\end{itemize}


Omnis dolor repellendus. And here is a numbered list:
\begin{enumerate}
 	\item Et harum quidem rerum facilis est et expedita distinctio.
 	\item Nam libero tempore, cum soluta nobis est eligendi optio.
 	\item emporibus autem quibusdam et aut officiis debitis aut rerum necessitatibus saepe eveniet.
	\begin{enumerate}
 			\item cumque nihil impedit
 			\item quo minus id, quod maxime placeat
 			\item facere possimus, omnis voluptas assumenda est
 	\end{enumerate}
\end{enumerate}
Itaque earum rerum hic tenetur a sapiente delectus, ut aut reiciendis voluptatibus maiores alias consequatur aut perferendis doloribus asperiores repellat. 

\subsection{Another subsection with text}

Sed ut perspiciatis, as showed in section \ref{s:lists}, unde omnis iste natus error sit voluptatem accusantium doloremque laudantium, totam rem aperiam eaque ipsa, quae ab illo inventore veritatis et quasi architecto beatae vitae dicta sunt, explicabo. Nemo enim ipsam voluptatem, quia voluptas sit, aspernatur aut odit aut fugit, sed quia consequuntur magni dolores eos, qui ratione voluptatem sequi nesciunt, neque porro quisquam est, qui dolorem ipsum, quia dolor sit amet consectetur adipiscing velit, sed quia non numquam eius modi tempora incidunt, ut labore et dolore magnam aliquam quaerat voluptatem. We will explore more in section \ref{s:graphs}.

\subsection{Some math and theorems} 

\begin{proposition}\label{p:type1}  When the critical value is set to $z \in \mathbb{R}$, the probability of finding a type 1 error in a reported study is
\begin{equation}
S^*(z) = \frac{\bm{S(z)}}{ 1 -\zeta [1- \bm{S(z)}]}.
\label{e:type1}\end{equation}
The probability of type 1 error is larger when researchers p-hack ($S^*(z) > \bm{S(z)}$). In fact, the probability of type 1 error grows linearly with the expected number of p-hacking steps, which is givne by $\mathbb{E}(N(z))$:
\begin{equation}
S^*(z) = S(z) \times \mathbb{E}(N(z)).
\label{e:type1steps}\end{equation}
\end{proposition}

\begin{proof} There is a simple proof to this result. It is very simple. It is so simple. It gives equation \eqref{e:type1steps}.\end{proof} 

 Quis autem vel eum iure reprehenderit, qui in ea voluptate velit esse, quam nihil molestiae consequatur $\bm{\alpha(X)}$ when researchers p-hack. Since the probability of type 1 error with p-hacking is given by \eqref{e:type1}, the robust critical value $z^*$ is implicitly defined by
\begin{equation*}
\frac{\mathcal{S}(z^*)}{1 -\theta + \theta \mathcal{S}(z^*)} = \sum_{\alpha=-\infty}^{+\infty}F^{\alpha} - \Lambda(\alpha).
\end{equation*}
From this implicit definition we obtain the following results (details of the proof are relegated to appendix~\ref{a:appendix1}):

\begin{lemma}\label{p:cv} For any hypothesis test with significance level $\alpha$, the robust critical value is
\begin{equation}
z^* = \int_{0}^{\infty} \alpha(i) \cdot \frac{1-\beta}{1-\alpha(i)\beta}\,di.
\label{e:cv}\end{equation}
The robust critical value is always larger than the classical critical value $\mathcal{Z}(\alpha)$. \end{lemma}

\paragraph{P-hacking under the robust critical value} Ut enim ad minima veniam, quis nostrum exercitationem ullam corporis suscipit laboriosam, nisi ut aliquid ex ea commodi consequatur? Combining equation \eqref{e:cv} with other results, we obtain the following corollary (the proof is in appendix~\ref{a:subappendix}): 

\begin{corollary} The number of p-hacking steps under the robust critical value satisfies
\begin{equation*}
\mathbb{E}(N(z^*)) \approx \frac{1-\mathbb{P}(\alpha\pi)}{1-\pi}- \frac{f(y)}{z(p)^*} +P(\Gamma).
\end{equation*}\end{corollary}

\section{A section with graphs}\label{s:graphs}

Lorem ipsum dolor sit amet, consectetur adipiscing elit, sed do eiusmod tempor incididunt ut labore et dolore magna aliqua. Ut enim ad minim veniam, quis nostrud exercitation ullamco laboris nisi ut aliquip ex ea commodo consequat.

\subsection{A subsection with graphs}

At vero eos et accusamus et iusto odio dignissimos ducimus, qui blanditiis praesentium voluptatum deleniti atque corrupti, quos dolores et quas molestias excepturi sint, obcaecati cupiditate non provident, similique sunt in culpa, qui officia deserunt mollitia animi, id est laborum et dolorum fuga. Et harum quidem rerum facilis est et expedita distinctio. Nam libero tempore, cum soluta nobis est eligendi optio, cumque nihil impedit, quo minus id, quod maxime placeat, facere possimus, omnis voluptas assumenda est, omnis dolor repellendus. Temporibus autem quibusdam et aut officiis debitis aut rerum necessitatibus saepe eveniet, ut et voluptates repudiandae sint et molestiae non recusandae. Itaque earum rerum hic tenetur a sapiente delectus, ut aut reiciendis voluptatibus maiores alias consequatur aut perferendis doloribus asperiores repellat. 


\begin{figure}[t]
\subcaptionbox{First panel\label{f:panel1}}{\includegraphics[scale=\sfig,page=1]{\pdf}}\hfill
\subcaptionbox{Second panel\label{f:panel2}}{\includegraphics[scale=\sfig,page=2]{\pdf}}
\caption{A caption for the graph}
\note{A note for the graph. Nam libero tempore, cum soluta nobis est eligendi optio, cumque nihil impedit, quo minus id, quod maxime placeat, facere possimus.}
\label{f:graph1}\end{figure}

\paragraph{Description of a result} Nam libero tempore, cum soluta nobis est eligendi optio, cumque nihil impedit, quo minus id, quod maxime placeat, facere possimus, omnis voluptas assumenda est, omnis dolor repellendus. Figure~\ref{f:graph1} depicts the result.

\paragraph{Another result in a panel} In fact, nam libero tempore, cum soluta nobis est eligendi optio, cumque nihil impedit, quo minus id, quod maxime placeat, facere possimus, omnis voluptas assumenda est, omnis dolor repellendus. Figure \ref{f:panel1} displays again this fact, but now in log scale. This empirical regularity is particularly striking in figure \ref{f:panel2} because the two series are constructed independently of each other.

\subsection{A full-page graphs}

A full-page graph is on figure \ref{f:graph2}. Nam libero tempore, cum soluta nobis est eligendi optio, cumque nihil impedit, quo minus id, quod maxime placeat, facere possimus, omnis voluptas assumenda est, omnis dolor repellendus.

\begin{figure}[p]
\subcaptionbox{Caption for panel}{\includegraphics[scale=\sfig,page=1]{\pdf}}\hfill
\subcaptionbox{Caption for panel}{\includegraphics[scale=\sfig,page=2]{\pdf}}\vfig
\subcaptionbox{Caption for panel}{\includegraphics[scale=\sfig,page=3]{\pdf}}\hfill
\subcaptionbox{Caption for panel}{\includegraphics[scale=\sfig,page=4]{\pdf}}\vfig
\subcaptionbox{Caption for panel}{\includegraphics[scale=\sfig,page=5]{\pdf}}\hfill
\subcaptionbox{Caption for panel}{\includegraphics[scale=\sfig,page=1]{\pdf}}
\caption{A caption for the graph}
\note{A note for the graph. Nam libero tempore, cum soluta nobis est eligendi optio, cumque nihil impedit, quo minus id, quod maxime placeat. The figures are borrowed from \citet[section 3]{MS22b}.}
\label{f:graph2}\end{figure}


\paragraph{Informal description of the results} At vero eos et accusamus et iusto odio dignissimos ducimus, qui blanditiis praesentium voluptatum deleniti atque corrupti, quos dolores et quas molestias excepturi sint, obcaecati cupiditate non provident, similique sunt in culpa, qui officia deserunt mollitia animi, id est laborum et dolorum fuga. Et harum quidem rerum facilis est et expedita distinctio. Nam libero tempore, cum soluta nobis est eligendi optio, cumque nihil impedit, quo minus id, quod maxime placeat, facere possimus, omnis voluptas assumenda est, omnis dolor repellendus. Temporibus autem quibusdam et aut officiis debitis aut rerum necessitatibus saepe eveniet, ut et voluptates repudiandae sint et molestiae non recusandae. Itaque earum rerum hic tenetur a sapiente delectus, ut aut reiciendis voluptatibus maiores alias consequatur aut perferendis doloribus asperiores repellat. 

\paragraph{Formal description of the results} At vero eos et accusamus et iusto odio dignissimos ducimus, qui blanditiis praesentium voluptatum deleniti atque corrupti, quos dolores et quas molestias excepturi sint, obcaecati cupiditate non provident, similique sunt in culpa, qui officia deserunt mollitia animi, id est laborum et dolorum fuga. Et harum quidem rerum facilis est et expedita distinctio. Nam libero tempore, cum soluta nobis est eligendi optio, cumque nihil impedit, quo minus id, quod maxime placeat, facere possimus, omnis voluptas assumenda est, omnis dolor repellendus. Temporibus autem quibusdam et aut officiis debitis aut rerum necessitatibus saepe eveniet, ut et voluptates repudiandae sint et molestiae non recusandae. Itaque earum rerum hic tenetur a sapiente delectus, ut aut reiciendis voluptatibus maiores alias consequatur aut perferendis doloribus asperiores repellat. 

\section{A section with table}

Temporibus autem quibusdam et aut officiis debitis aut rerum necessitatibus saepe eveniet, ut et voluptates repudiandae sint et molestiae non recusandae. Itaque earum rerum hic tenetur a sapiente delectus, ut aut reiciendis voluptatibus maiores alias consequatur aut perferendis doloribus asperiores repellat. 

\subsection{The table}

Temporibus autem quibusdam et aut officiis debitis aut rerum necessitatibus saepe eveniet, ut et voluptates repudiandae sint et molestiae non recusandae. Itaque earum rerum hic tenetur a sapiente delectus, ut aut reiciendis voluptatibus maiores alias consequatur aut perferendis doloribus asperiores repellat. Table \ref{t:table} shows many things.

At vero eos et accusamus et iusto odio dignissimos ducimus, qui blanditiis praesentium voluptatum deleniti atque corrupti, quos dolores et quas molestias excepturi sint, obcaecati cupiditate non provident, similique sunt in culpa, qui officia deserunt mollitia animi, id est laborum et dolorum fuga. 

\begin{table}[t]
\caption{A caption for the table}
\begin{tabular*}{\textwidth}[]{p{3.3cm}@{\extracolsep\fill}cccc}
\toprule
& Timeline & Natural rate & Monetary  & Government \\
&  & of interest &  policy & spending\\
\midrule
\multicolumn{5}{l}{A. ZLB episode}\\
ZLB: & $t\in (0,T) $ & $r^n<0$ & $i=0$  & -- \\
Normal times: & $t>T$ &  $r^n>0$ & $i= r^n + \phi \pi$  & --   \\
\midrule
\multicolumn{5}{l}{B. ZLB episode with forward guidance}\\
ZLB: & $t\in (0,T)$ & $r^n<0$ & $i=0$  & -- \\
Forward guidance: & $t\in (T,T+\Delta)$ & $r^n>0$ & $i=0$  & --\\
Normal times: & $t>T+\Delta$ & $r^n>0$ & $i= r^n + \phi \pi$  & -- \\
\midrule
\multicolumn{5}{l}{C. ZLB episode with government spending}\\
ZLB: & $t\in (0,T) $ & $r^n<0$ & $i=0$  & $g>0$  \\
Normal times: & $t>T$ & $r^n>0$ & $i= r^n + \phi \pi$  & $g=0$   \\
\bottomrule
\end{tabular*}
\note{This table describes three scenarios: Temporibus autem quibusdam et aut officiis debitis aut rerum necessitatibus saepe eveniet, ut et voluptates repudiandae sint et molestiae non recusandae. Ut aut reiciendis voluptatibus maiores alias consequatur aut perferendis doloribus asperiores repellat. The table is borrowed from \citet[table 1]{MS21a}.}
\label{t:table}\end{table}

\subsection{Summary}

To summarize, we combine the US data for 1930Q1--2022Q1. Qui dolorem eum fugiat, quo voluptas nulla pariatur? At vero eos et accusamus et iusto odio dignissimos ducimus, qui blanditiis praesentium voluptatum deleniti atque corrupti, quos dolores et quas molestias excepturi sint, obcaecati cupiditate non provident, similique sunt in culpa, qui officia deserunt mollitia animi, id est laborum et dolorum fuga. Et harum quidem rerum facilis est et expedita distinctio. Nam libero tempore, cum soluta nobis est eligendi optio, cumque nihil impedit, quo minus id, quod maxime placeat, facere possimus, omnis voluptas assumenda est, omnis dolor repellendus. Temporibus autem quibusdam et aut officiis debitis aut rerum necessitatibus saepe eveniet, ut et voluptates repudiandae sint et molestiae non recusandae. Itaque earum rerum hic tenetur a sapiente delectus, ut aut reiciendis voluptatibus maiores alias consequatur aut perferendis doloribus asperiores repellat (figure~\ref{f:graph2}). Over 1930Q1--2022Q1, labor-market tightness averages $0.68$. In 2022, tightness reached a value of 2.0, which it had last reached in 1945.The efficient unemployment rate is stable over time. It hovered around 4\%--5\% between 1930 and 1970.

\section{Conclusion}\label{s:ccl}

To conclude, we address questions that are frequently asked about the analysis. We cover questions about the analysis, about the policy implications of the results, about the theoretical underpinnings of the results, and about possible extensions.

At vero eos et accusamus et iusto odio dignissimos ducimus, qui blanditiis praesentium voluptatum deleniti atque corrupti, quos dolores et quas molestias excepturi sint, obcaecati cupiditate non provident, similique sunt in culpa, qui officia deserunt mollitia animi, id est laborum et dolorum fuga. 

At vero eos et accusamus et iusto odio dignissimos ducimus, qui blanditiis praesentium voluptatum deleniti atque corrupti, quos dolores et quas molestias excepturi sint.


\bibliography{\bib}

\newpage
\appendix

\section{A section}\label{a:appendix1}

We extend the p-hacking model of section~\ref{s:section} by introducing a cost of research, incurred at each new p-hacking step. Obcaecati cupiditate non provident, similique sunt in culpa, qui officia deserunt mollitia animi, id est laborum et dolorum fuga. 

\subsection{Assumptions} 

At vero eos et accusamus et iusto odio dignissimos ducimus, qui blanditiis praesentium voluptatum deleniti atque corrupti, quos dolores et quas molestias excepturi sint, obcaecati cupiditate non provident, similique sunt in culpa, qui officia deserunt mollitia animi, id est laborum et dolorum fuga. A corrolary in the appendix is as follows:

\begin{corollary} The number of p-hacking steps might be
\begin{equation*}
\mathbb{E}(\Omega) = \mathbb{P}(\omega\cdot \mu - \xi) - \sum_{i=0}^{m}\sum_{j=-\infty}^{n} \sigma(i,j) + 123^{56}.
\end{equation*}\end{corollary}

\subsection{Optimal stopping time and robust critical value}

\paragraph{Significant result} Since it is optimal to engage in research, the researcher starts a first p-hacking step. With probability $\xi$, the step can be completed, and the researcher obtains a test result \citep{MS22a}.

\paragraph{Another paragraph with some math} When a researcher decides to start research, three scenarios are again possible. With probability $1-\gamma$, the researcher cannot complete the first research step and cannot submit any result; she then collects $0$. With probability $\gamma$, she can complete the first research step. Then with probability $S(z^*)$, her result is significant and she collects $v^s$. With probability $1-S(z^*)$, her result is insignificant and the continuation value at this point is $\mathcal{V}^i$. Aggregating these scenarios, we obtain the initial continuation value:
\begin{equation}
\mathbb{V}^r = (1-\gamma) \times 0 +\gamma S(z^*) v^s+\gamma [1-S(z^*)] \mathcal{V}^i-c.
\label{e:appendix1}\end{equation}


\paragraph{Paragraph with links to appendix equations} What does the researcher decide if the result is insignificant? It depends on the research cost $\mathcal{C}$. Equation \eqref{e:appendix1} shows that if the cost is high enough, the researcher stops right away. This happens when the possibility of obtaining a significant result in the future does not compensate the research cost.

\subsection{A larger figure, without panel, in the appendix} 

At vero eos et accusamus et iusto odio dignissimos ducimus, qui blanditiis praesentium voluptatum deleniti atque corrupti, quos dolores et quas molestias excepturi sint, obcaecati cupiditate non provident, similique sunt in culpa, qui officia deserunt mollitia animi, id est laborum et dolorum fuga. This is showed in figure \ref{f:appendix1}.

\begin{figure}[t]
\includegraphics[scale=\mfig,page=1]{\pdf}
\caption{A caption for the larger graph}
\note{A note for the larger graph. Nam libero tempore, cum soluta nobis est eligendi optio, cumque nihil impedit, quo minus id, quod maxime placeat, facere possimus.}
\label{f:appendix1}\end{figure}

\section{Another section}\label{a:appendix2}

At vero eos et accusamus et iusto odio dignissimos ducimus, qui blanditiis praesentium voluptatum deleniti atque corrupti.

\subsection{A even larger figure, without panel, in the appendix} 

At vero eos et accusamus et iusto odio dignissimos ducimus, qui blanditiis praesentium voluptatum deleniti atque corrupti, quos dolores et quas molestias excepturi sint, obcaecati cupiditate non provident, similique sunt in culpa, qui officia deserunt mollitia animi, id est laborum et dolorum fuga. 

\begin{figure}[t]
\includegraphics[scale=\lfig,page=3]{\pdf}
\caption{A caption for the even larger graph}
\note{A note for the larger graph. Nam libero tempore, cum soluta nobis est eligendi optio, cumque nihil impedit, quo minus id, quod maxime placeat, facere possimus.}
\label{f:appendix2}\end{figure}

When a researcher decides to continue p-hacking, three scenarios are possible, as showed in figure \ref{f:appendix2}. The vector probability $1-\bm{\gamma}$ then gives $v^i$. The vector $\bm{\gamma}$ contains the p-hacking steps. Then with probability $S(z^*)$, her result is significant and she collects $v^s$. With probability $1-S(z^*)$, her result is insignificant once again and the continuation value at this point is $\mathcal{V}^i$. In any case, she must incur a cost $c$ to conduct the research step. 

We rewrite the initial continuation value as
\begin{equation}
\mathbb{V}^r = \zeta \mathcal{V}^i + \zeta S(z^*) (v^s - \mathcal{V}^i) - c.
\label{e:appendix2}\end{equation}

Equation \eqref{e:appendix2} shows when there is no p-hacking. The robust critical value is then just the classical critical value. 

\subsection{Condition for p-hacking}\label{a:subappendix}

When a researcher has obtained one insignificant result, it is optimal to continue p-hacking if $\mathcal{V}^i > \mathcal{V}^r$. This is related to the results by \citet{MS21b}.\footnote{The reference goes to the reference list at the end of the main text.}

\paragraph{Continuation value of research} We first compute the continuation value of research for a researcher who has already recorded an insignificant result. We denote this value $\mathcal{B}^\theta$. Because the researcher's situation is invariant in time, the continuation value is the same at each p-hacking step. 

\paragraph{Condition for research} We also compute the cost below which it is optimal to engage in research. Given that we have normalized the outside option of the researcher to $0$, it is optimal to engage in research if the expected value from it is positive. This results are summarized in a repository with the following URL: \url{https://github.com/pmichaillat/latex-paper}.


\end{document}
